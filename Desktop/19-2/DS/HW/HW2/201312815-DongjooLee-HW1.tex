\documentclass[12pt,a4paper]{article}
\usepackage{amsmath,amssymb,fullpage,graphicx,enumerate,enumitem,bm}
\let\hat\widehat
\let\tilde\widetilde

\usepackage{amssymb}
\usepackage{amsmath}
\usepackage{epsfig, graphics}
\graphicspath{{./assets/}}
\usepackage{latexsym}
\usepackage{fullpage}
\usepackage{bm}
\usepackage[parfill]{parskip}
%\usepackage{subfigure}
\usepackage{subcaption}
% TODO : Fil your personal package here if needed
\usepackage{caption}
\usepackage{float}


%%%% new version of enumerate with less spacing
\newenvironment{enum}{
    \begin{enumerate}
        \setlength{\itemsep}{1pt}
            \setlength{\parskip}{0pt}
            \setlength{\parsep}{0pt}
    }{\end{enumerate}}

    % TODO : fill your custom commands here if needed
\newcommand{\infint}{\int_{-\infty}^{\infty}}
\newcommand{\Ab}{\bm{A}}

\begin{document}

% TODO : Fill your personal information here
\begin{center}
    {\bf\large Homework 2} \\
    {\bf\large M1522.000900 Data Structure (2019 Fall)} \\
    2013-12815 Dongjoo Lee \\
\end{center}

%%%%%%%%%%%%%%%%%%%%%%%%%%%%%%%%%%%%%%%%%%%%%%
% Q1
%%%%%%%%%%%%%%%%%%%%%%%%%%%%%%%%%%%%%%%%%%%%%%
\section{Q1}
Growth rate has an order like $n!>a^n>n^k>logn>c$ from fastest. By this, growth rate above has an order in $\mathbf{n!>3\cdot2^n>6n^2>20n>log_2 n>log_2 log_2 n>20n}$.

%%%%%%%%%%%%%%%%%%%%%%%%%%%%%%%%%%%%%%%%%%%%%%
% Q2
%%%%%%%%%%%%%%%%%%%%%%%%%%%%%%%%%%%%%%%%%%%%%%
\section{Q2}
\begin{enumerate}[label=(\arabic*)]
    \item
        $f(n)=logn^2=2logn$\\
        $g(n)=logn+5$\\
        For $n_0=10^6$ and $c=1$, $f(n)\geq  c\cdot g(n)$, $\forall n>n_0$\\
        $\therefore f(n)=\Omega(g(n))$\\
        And for $n_0=0$ and $c=2$, $f(n)\leq  c\cdot g(n)$, $\forall n>n_0$\\
        $\therefore f(n)=O(g(n))$\\
        $\mathbf{\therefore f(n)=\theta(g(n))}$

    \item
        $f(n)=\sqrt{n}=n^\frac{1}{2}$\\
        $g(n)=log n^2$\\
        For $n_0=10^2$ and $c=1$, $f(n)\geq  c\cdot g(n)$, $\forall n>n_0$\\
        $\mathbf{\therefore f(n)=\Omega(g(n))}$

    \item
        $f(n)=n$\\
        $g(n)=log^2 n=(logn)^2$\\
        At $n>10$, $logn<n^\frac{1}{2}$, thus, $(logn)^2<(n^\frac{1}{2})^2=n$\\
        For $n_0=10$, and $c=1$, $f(n)\geq  c\cdot g(n)$, $\forall n>n_0$\\
        $\mathbf{\therefore f(n)=\Omega(g(n))}$

    \item
        $f(n)=logn^2=2logn$\\
        $g(n)=log^2n=(logn)^2$\\
        At $n>10$, $logn>1$ and $(logn)^2>logn$\\
        For $n_0=10$ and $c=2$, $f(n)\leq  c\cdot g(n)$, $\forall n>n_0$\\
        $\mathbf{\therefore f(n)=O(g(n))}$
\end{enumerate}

%%%%%%%%%%%%%%%%%%%%%%%%%%%%%%%%%%%%%%%%%%%%%%
% Q3
%%%%%%%%%%%%%%%%%%%%%%%%%%%%%%%%%%%%%%%%%%%%%%
\section{Q3}
\begin{enumerate}[label=(\arabic*)]
    \item
        \begin{enumerate}
            \item
                Before the 1st for loop, time complexity is $\theta(1)$.
                For the 1st for loop, all the comparison and calculation and assignment occur constant times at each repetitions.
                So, 1st for loop's time complexity is $\theta(N)$.
                Similarly, 2nd for loop's time complexity is $\theta(M)$.\\
                \textbf{$\therefore$ Total time complexity is $\mathbf{\theta(N+M)}$.}
                %comparison between $i, N$ and incremental  time complexity is \theta(n).
                %And for the 2nd for loop, time complexity is 
            \item
                Before the 1st for loop, memory allocating occurs for a and b. So, its space complexity is $\theta(1)$.
                For the 1st for loop, memory allocating occurs only for i and a at each repetitions. So, its space complexity is $\theta(1)$.
                Similarly, 2nd for loop's space complexity is also $\theta(1)$.\\
                \textbf{$\therefore$ Total space complexity is $\mathbf{\theta(1)}$.}
        \end{enumerate}
    \item
        Before the 1st for loop, time complexity is $\theta(1)$.
        For the outer for loop, $i$ increases by 1, so it is repeated $c_1\cdot n$ times.
        For the inner for loop, $j$ increases in $2^1, 2^2, \cdots, 2^(lgn)$, so it is repeated $c_2\cdot lgn$ times.
        Therefore, the whole repetition occurs $c_1\cdot c_2\cdot n\cdot lgn$ times.\\
        \textbf{$\therefore$ Total time complexity is $\mathbf{\theta(n\cdot lgn)}$.}

    \item
        An algorithm $X$ is asymptotically more efficient than $Y$ means $X$ will always be a better choice for large inputs.\\
        \textbf{$\therefore$ (b)}

\end{enumerate}

%%%%%%%%%%%%%%%%%%%%%%%%%%%%%%%%%%%%%%%%%%%%%%
% Q4
%%%%%%%%%%%%%%%%%%%%%%%%%%%%%%%%%%%%%%%%%%%%%%
\section{Q4}
\begin{enumerate}[label=(\arabic*)]
    \item
        $T(n)+1=3\cdot(T(n-1)+1)$\\
        $T(n)+1=3^{n-1}\cdot(T(0)+1)$\\
        $\therefore T(n)=3^{n-1}\cdot(T(0)+1)-1 = 3^{n-1}\cdot T(0)+3^{n-1}-1$\\
        Thus, for $c=T(0)+1$ and $n_0=0$, $T(n)\leq c\cdot 3^{n-1}$, $\forall n>n_0$\\
        $\mathbf{\therefore T(n)=O(3^n)}$

    \item
        Let's assume that the new machine $Y$ has similar algorithm with $X$.
        $X$ takes $t$ seconds for $n$ inputs. As $Y$ is 27 times faster than $X$, Y takes $\frac{1}{27}t$ for the same $n$ inputs.\\
        $c\cdot 3^{n-1} = 27\cdot c'\cdot 3^{n-1}=t$\\
        $c'=\frac{1}{27}c$\\
        $\therefore$ Time complexity of $Y$, $Y(n)=c\cdot 3^{n-4}$\\
        By the result, \textbf{$\mathbf{Y}$ can handle $\mathbf{n+3}$ inputs} while $X$ handles $n$ inputs in the same $t$.

\end{enumerate}

\section{Q5}
\begin{enumerate}[label=(\arabic*)]
    \item
        In the function powerN, comparison occurs 1 time in the 1st line.
        At the 2nd line, function call occurs.
        If time complexity of function in Figure 1. is $T(n)$, then, \\
        $T(n)=1+T(n-1)$\\
        \\
        $\therefore$ If $n$ is large enough, $\mathbf{T(n)=\theta(n)}$.

    \item
        If time complexity of function in Figure 2. is $S(n)$, then,\\
        \[
            S(n) = \left.
            \begin{cases}
                1 & \text{for } n=0\\
                1+S(\frac{n}{2}) & \text{for } n=2^k\\
                1+S(\frac{n-1}{2}) & \text{for } n=2^k-1\\
            \end{cases}
            \right\}
        \]
        $\therefore$ If $n$ is large enough, $\mathbf{S(n)=\theta(lgn)}$.
        
\end{enumerate}

\end{document}
